\documentclass{beamer}

\usepackage[utf8]{inputenc}
\usepackage{graphicx}

\usepackage{listings}
\usepackage{subfiles}

\title{Introduction to elm}
\author{Thomas Vandeven}
\institute{London Functional Programmers}
\date{\today}
\usetheme{Boadilla}

\newenvironment{codeexample}{
    \begin{example}}{\end{example}}
\newcommand{\rar}{\rightarrow}
\newcommand{\bigpicture}{
    \frame{\tableofcontents[currentsection,hideallsubsections]}
}

\begin{document}
    \frame{\titlepage}

% -- OUTLINE

    \begin{frame}
        \frametitle{Outline}
        \tableofcontents[hideallsubsections]
    \end{frame}

%-- Intro
    \section{Introduction}
    \bigpicture
    \begin{frame}
        \subsection{Introduction: What is elm}
        \frametitle{Introducing elm}
        \begin{itemize}
        \item{Elm is a functional programmming language designed 
        by Evan Czaplicki in 2012 for making GUIs.}

        \item{It transpiles to javascript and is used to
        write web applications.}
        \end{itemize}
    \end{frame}

%-- WHY ELM
    \begin{frame}
        \frametitle{Why Elm}

        Elm is about:

        \begin{itemize}
            \item{No runtime errors in practice. No null.}
            \item{Friendly error messages that make development easier}
            \item{Well-architected code that stays well-architected as your app grows.}
            \item{Automatically enforced semantic versioning for all Elm packages.}
        \end{itemize}

    \end{frame}

    \section{Language Basics}
    \bigpicture
%-- ELM BASICS

    %-- TYPES
    
    \subfile{types}
    \subfile{types_continued}
    \subfile{types_list}
    \subfile{functions}
    \subfile{records}
    \subfile{custom_types}
    \subfile{pattern_matching}

    \section{The Elm Architecture}
    \bigpicture
    \subfile{tea_overview}
    \subfile{side_effects}
    \subfile{tea_image}


    \section{Example Projects}
    \bigpicture
    \subfile{counter}
    \subfile{cat_generator}
\end{document}